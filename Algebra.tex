\documentclass[12pt,a4paper]{report}
\oddsidemargin=-40pt
\topmargin=-35pt
\usepackage{amsmath}
\usepackage{hyperref}
\hypersetup{
colorlinks=true, %set true if you want colored links
linktoc=all,     %set to all if you want both sections and subsections linked
linkcolor=blue,  %choose some color, e.g. blue, if you want links to stand out
}

\begin{document}

\chapter{Definition}
\section{Special Number}
Complex number:
\begin{itemize}
\item $0$
\item $-1$
%\item $\tau = 2\pi$
\item $\tau = \frac{\pi}{2}$
\item $i$
\end{itemize}
Infinity:
\begin{itemize}
%\item $\infty$
\item $-\infty$
\end{itemize}

\section{Definition:Addition}
\label{Definition:Addition}
Closure: If $x$ and $y$ is complex number, then $x+y$ is complex number.

\section{Definition:Exponential Function}
\label{Definition:Exponential_Function}
Closure: If $x$ is complex number, then $\exp(x)$ is complex number. \\

\section{Definition:Logarithmic Function}
\label{Definition:Logarithmic_Function}
For $x \neq -\infty$ \\
If $x$ is complex number and $x \neq 0$, then $\ln(x)$ is complex number. \\

\section{Definition:Variable and Constant}
x is complex number and $-\infty$. c is integer constant.

\chapter{Axiom}
\section{Axiom:Associativity}
\label{Axiom:Associativity}
\begin{align*}
(x+y)+z = x+(y+z)
\end{align*}

\section{Axiom:Commutativity}
\label{Axiom:Commutativity}
\begin{align*}
x+y = y+x
\end{align*}

\section{Axiom:Additive Identity}
\label{Axiom:Additive_Identity}
\begin{align*}
x+0 = 0+x = x
\end{align*}
Simplification:
\begin{align*}
x+0 = 0+x = x \\
c+0 = 0+c = c && \text{(basic arithmetic)}
\end{align*}

\section{Axiom:Addition of Infinity}
\label{Axiom:Addition_of_Infinity}
For $x$ is complex number,
\begin{align*}
\infty + x &= \infty && \text{not defined yet} \\
-\infty + x &= -\infty \\
\infty + \infty &= \infty && \text{not defined yet} \\
(-\infty) + (-\infty) &= -\infty \\
\infty + (-\infty) &&& \text{not defined} \\
\end{align*}
Simplification:
\begin{align*}
x-\infty = -\infty+x = -\infty \\
c-\infty = -\infty+c = -\infty && \text{(basic arithmetic)}
\end{align*}

\section{Definition:Multiplicative Identity}
\label{Definition:Multiplicative_Identity}
\begin{align*}
1 = \exp (0)
\end{align*}
Simplification:
\begin{align*}
\exp (0) = 1
\end{align*}

\section{Axiom:Additive Inverse}
\label{Axiom:Additive_Inverse}
\begin{align*}
1 + (-1) = 0
\end{align*}
Simplification:
\begin{align*}
1 + (-1) = 0  && \text{(basic arithmetic)}
\end{align*}

\section{Definition:Multiplication}
\label{Definition:Multiplication}
For $x \neq -\infty$ and $y \neq -\infty$,
\begin{align*}
x \cdot y = \exp(\ln(x) + \ln(y))
\end{align*}

\section{Axiom:Distributivity}
\label{Axiom:Distributivity}
\begin{align*}
(x+y) \cdot z = x \cdot z + y \cdot z
\end{align*}
Expansion: (good for intermediate step for term cancellation)
\begin{align}
\exp(\ln(x+y) + z) = \exp(\ln(x) + z) + \exp(\ln(y) + z) \label{Implementation:Expansion}
\end{align}
Factorization: (good for the final step)
\begin{align}
\exp(\ln(x) + z) + \exp(\ln(y) + z) = \exp(\ln(x+y) + z) \label{Implementation:Factorization}
\end{align}
Simplification: \\
$c_1 \exp(x) + c_2 \exp(x) = (c_1+c_2) \exp(x)$
\begin{align}
\exp(\ln(c_1) + x) + \exp(\ln(c_2) + x) = \exp(\ln(c_1+c_2) + x) \label{Implementation:Summation_of_Term_1}
\end{align}
Summation of the same term: \\
$\exp(x) + y \exp(x) = (1+y) \exp(x)$
\begin{align}
\exp(x) + \exp(\ln(y) + x) = \exp(\ln(y+1) + x) \label{Implementation:Summation_of_Term_2}
\end{align}
$\exp(x) + c \exp(x) = (1+c) \exp(x)$
\begin{align}
\exp(x) + \exp(\ln(c) + x) = \exp(\ln(c+1) + x) \label{Implementation:Summation_of_Term_3}
\end{align}
$x + y x = (1+y) x$, do this simplification before expansion
\begin{align}
x + \exp(\ln(y) + \ln(x)) &= \exp(\ln(y+1) + \ln(x)) \label{Implementation:Summation_of_Term_4}
\end{align}
$x + c x = (1+c) x$, do this simplification before expansion
\begin{align}
x + \exp(\ln(c) + \ln(x)) &= \exp(\ln(c+1) + \ln(x)) \label{Implementation:Summation_of_Term_5}
\end{align}
$x + x = 2x$, do this simplification before expansion
\begin{align}
x + x &= \exp(\ln(2) + \ln(x)) \label{Implementation:Summation_of_Term_6}
\end{align}
For example,
\begin{align*}
& 6(x+y) + 2(x+y) - 5y - 3y \\
&= 8(x+y)- 8y &= 6x +6y + 2x +2y - 5y - 3y \\
&=8x+8y-8y &= 8x \\
&=8x
\end{align*}

\section{Axiom:Inverse Function 1}
\label{Axiom:Inverse_Function_1}
For $x \neq -\infty$,
\begin{align*}
\exp (\ln (x)) &= x \\
\end{align*}
Simplification:
\begin{align*}
\exp (\ln (x)) &= x \\
\end{align*}

\section{Axiom:Inverse Function 2}
\label{Axiom:Inverse_Function_2}
\begin{align*}
\exp (\ln (\exp (x)) + y) &= \exp(x + y) \\
\end{align*}
Simplification:
\begin{align*}
\exp (\ln (\exp (x)) + y) &= \exp(x + y) \\
\end{align*}

\section{Axiom:Exponentiation of Infinity}
\label{Axiom:Exponentiation_of_Infinity}
\begin{align*}
\exp (-\infty) &= 0 \\
\exp (\infty) = \pm \infty && \text{not defined}
\end{align*}
Simplification:
\begin{align*}
\exp (-\infty) &= 0 \\
\end{align*}

\section{Definition:Multiplicative Inverse}
\label{Definition:Multiplicative_Inverse}
For $x \neq 0$ and $x \neq -\infty$,
\begin{align*}
x^{-1} = \exp(-\ln(x))
\end{align*}

\section{Definition:Fraction}
\label{Definition:Fraction}
For $x \neq -\infty$, $y \neq 0$ and $y \neq -\infty$,
\begin{align*}
\frac{x}{y} = x \cdot y^{-1}
\end{align*}

\section{Axiom:Exponentiation of tau and i}
\label{Axiom:Exponentiation_of_tau_and_i}
\begin{align*}
\exp (2 \cdot \tau \cdot i) &= -1 \\
\exp (\tau \cdot i) &= i \\
\end{align*}
Simplification:
\begin{align*}
\exp \big(\exp[ (\ln(2) + \ln(\tau) + \ln(i) ] \big) &= -1 \\
\exp \big(\exp[ (\ln(\tau) + \ln(i) ] \big) &= i \\
\end{align*}

\chapter{Field Theorem}
\section{Theorem:Multiplicative Associativity}
\label{Theorem:Multiplicative_Associativity}
For $x \neq -\infty$ and $y \neq -\infty$ and and $z \neq -\infty$,
\begin{align*}
(x \cdot y) \cdot z
&= \exp(\ln(x) + \ln(y)) \cdot z
&& \text{Definition \ref{Definition:Multiplication}} \\
&= \exp(\ln(\exp(\ln(x) + \ln(y))) + \ln(z))
&& \text{Definition \ref{Definition:Multiplication}} \\
&= \exp((\ln(x) + \ln(y)) + \ln(z))
&& \text{Axiom \ref{Axiom:Inverse_Function_2}} \\
&= \exp(\ln(x) + (\ln(y) + \ln(z)))
&& \text{Axiom \ref{Axiom:Associativity}} \\
&= \exp(\ln(x) + \ln(\exp(\ln(y) + \ln(z))))
&& \text{Axiom \ref{Axiom:Inverse_Function_2}} \\
&= x \cdot \exp(\ln(y) + \ln(z))
&& \text{Definition \ref{Definition:Multiplication}} \\
&= x \cdot (y \cdot z)
&& \text{Definition \ref{Definition:Multiplication}} \\
\end{align*}
Reduction:
\begin{align*}
\exp(\ln(x) + \ln(y) + \ln(z))
\end{align*}

\section{Theorem:Multiplicative Commutativity}
\label{Theorem:Multiplicative_Commutativity}
For $x \neq -\infty$ and $y \neq -\infty$,
\begin{align*}
x \cdot y
&= \exp(\ln(x) + \ln(y))
&& \text{Definition \ref{Definition:Multiplication}} \\
&= \exp(\ln(y) + \ln(x))
&& \text{Axiom \ref{Axiom:Commutativity}} \\
&= y \cdot x
&& \text{Definition \ref{Definition:Multiplication}} \\
\end{align*}
Reduction:
\begin{align*}
\exp(\ln(x) + \ln(y))
\end{align*}

\section{Theorem:Logarithm of One}
\label{Theorem:Logarithm_of_One}
\begin{align*}
\exp(\ln(1) + x)
&= \exp(\ln(\exp(0)) + x)
&& \text{Definition \ref{Definition:Multiplicative_Identity}} \\
&= \exp(0 + x)
&& \text{Axiom \ref{Axiom:Inverse_Function_2}} \\
\end{align*}
Simplification:
\begin{align*}
\exp(\ln(1) + x) &= \exp(0 + x) \\
\end{align*}

\section{Theorem:Logarithm of Zero}
\label{Theorem:Logarithm_of_Zero}
\begin{align*}
\exp(\ln(0) + x)
&= \exp(\ln(\exp(-\infty)) + x)
&& \text{Axiom \ref{Axiom:Exponentiation_of_Infinity}} \\
&= \exp(-\infty + x)
&& \text{Axiom \ref{Axiom:Inverse_Function_2}} \\
\end{align*}
Simplification:
\begin{align*}
\exp(\ln(0) + x) &= \exp(-\infty + x) \\
\end{align*}

\section{Theorem:Logarithm of Negative One}
\label{Theorem:Logarithm_of_Negative_One}
\begin{align*}
\exp(\ln(-1) + x)
&= \exp(\ln(\exp(2 \cdot \tau \cdot i)) + x)
&& \text{Axiom \ref{Axiom:Exponentiation_of_tau_and_i}} \\
&= \exp(2 \cdot \tau \cdot i + x)
&& \text{Axiom \ref{Axiom:Inverse_Function_2}} \\
\end{align*}
Simplification:
\begin{align}
\exp(\ln(-1) + x) &= \exp(2 \cdot \tau \cdot i + x) && \text{(good for intermediate step for term cancellation)} \label{Implementation:Logarithm_of_Negative_One_1} \\
\exp(2 \cdot \tau \cdot i + x) &= \exp(\ln(-1) + x) && \text{(good for the final step)} \label{Implementation:Logarithm_of_Negative_One_2}
\end{align}

\section{Theorem:Logarithm of i}
\label{Theorem:Logarithm_of_i}
\begin{align*}
\exp(\ln(i) + x)
&= \exp(\ln(\exp(\tau \cdot i)) + x)
&& \text{Axiom \ref{Axiom:Exponentiation_of_tau_and_i}} \\
&= \exp(\tau \cdot i + x)
&& \text{Axiom \ref{Axiom:Inverse_Function_2}} \\
\end{align*}
Simplification:
\begin{align}
\exp(\ln(i) + x) &= \exp(\tau \cdot i + x) && \text{(good for intermediate step for term cancellation)} \label{Implementation:Logarithm_of_i_1} \\
\exp(\tau \cdot i + x) &= \exp(\ln(i) + x) && \text{(good for the final step)} \label{Implementation:Logarithm_of_i_2}
\end{align}

\section{Theorem:Multiplicative Identity}
\label{Theorem:Multiplicative_Identity}
For $x \neq -\infty$,
\begin{align*}
1 \cdot x
&= \exp(\ln(1) + \ln(x))
&& \text{Definition \ref{Definition:Multiplication}} \\
&= \exp(0 + \ln(x))
&& \text{Theorem \ref{Theorem:Logarithm_of_One}} \\
&= \exp(\ln(x))
&& \text{Axiom \ref{Axiom:Additive_Identity}} \\
&= x
&& \text{Axiom \ref{Axiom:Inverse_Function_1}} \\
\end{align*}
Auto Simplification

\section{Theorem:Multiplication of Zero}
\label{Theorem:Multiplication_of_Zero}
For $x \neq -\infty$,
\begin{align*}
0 \cdot x
&= \exp(\ln(0) + \ln(x))
&& \text{Definition \ref{Definition:Multiplication}} \\
&= \exp(-\infty + \ln(x))
&& \text{Theorem \ref{Theorem:Logarithm_of_Zero}} \\
&= \exp(-\infty)
&& \text{Axiom \ref{Axiom:Addition_of_Infinity}} \\
&= 0
&& \text{Axiom \ref{Axiom:Exponentiation_of_Infinity}} \\
\end{align*}
Auto Simplification

\section{Theorem:Summation of Term 1}
\label{Theorem:Summation_of_Term_1}
\begin{align*}
x + y \cdot x
&= 1 \cdot x + y \cdot x
&& \text{Theorem \ref{Theorem:Multiplicative_Identity}} \\
&= (1+y) \cdot x
&& \text{Axiom \ref{Axiom:Distributivity}} \\
\end{align*}
Simplification: (same as Implementation \ref{Implementation:Summation_of_Term_4} and \ref{Implementation:Summation_of_Term_5})
\begin{align*}
x + \exp(\ln(y) + \ln(x)) &= \exp(\ln(y+1) + \ln(x)) \\
\end{align*}

\section{Definition:Additive Inverse}
\label{Definition:Additive_Inverse}
\begin{align*}
-x = (-1) \cdot x
\end{align*}

\section{Theorem:Additive Inverse}
\label{Theorem:Additive_Inverse}
For $x \neq -\infty$,
\begin{align*}
x + (-x)
&= x + (-1) \cdot x
&& \text{Definition \ref{Definition:Additive_Inverse}} \\
&= (1 + (-1)) \cdot x
&& \text{Theorem \ref{Theorem:Summation_of_Term_1}} \\
&= 0 \cdot x
&& \text{Axiom \ref{Axiom:Additive_Inverse}} \\
&= 0
&& \text{Theorem \ref{Theorem:Multiplication_of_Zero}} \\
\end{align*}
Auto Simplification

\section{Definition:Two}
\label{Definition:Two}
\begin{align*}
2 &= 1+1 \\
\end{align*}

\section{Theorem:Summation of Term 3}
\label{Theorem:Summation_of_Term_3}
\begin{align*}
x + x
&= 1 \cdot x + 1 \cdot x
&& \text{Theorem \ref{Theorem:Multiplicative_Identity}} \\
&= (1+1) \cdot x
&& \text{Axiom \ref{Axiom:Distributivity}} \\
&= 2 \cdot x
&& \text{Definition \ref{Definition:Two}} \\
\end{align*}
Simplification: (same as Implementation \ref{Implementation:Summation_of_Term_6})
\begin{align*}
x + x &= \exp(\ln(2) + \ln(x)) \\
\end{align*}

\section{Theorem:Multiplicative Inverse}
\label{Theorem:Multiplicative_Inverse}
For $x \neq 0$ and $x \neq -\infty$,
\begin{align*}
x \cdot x^{-1}
&= x \cdot \exp(-\ln(x))
&& \text{Definition \ref{Definition:Multiplicative_Inverse}} \\
&= \exp( \ln(x) + \ln(\exp(-\ln(x))) )
&& \text{Definition \ref{Definition:Multiplication}} \\
&= \exp( \ln(x) + (-\ln(x)) )
&& \text{Axiom \ref{Axiom:Inverse_Function_2}} \\
&= \exp(0)
&& \text{Theorem \ref{Theorem:Additive_Inverse}} \\
&= 1
&& \text{Definition \ref{Definition:Multiplicative_Identity}} \\
\end{align*}
Auto Simplification

\section{Theorem:Multiplication of Positive and Negative}
\label{Theorem:Multiplication_of_Positive_and_Negative}
For $x \neq -\infty$,
\begin{align*}
x \cdot (-x)
&= x \cdot ((-1) \cdot x)
&& \text{Definition \ref{Definition:Additive_Inverse}} \\
&= (x \cdot (-1)) \cdot x
&& \text{Theorem \ref{Theorem:Multiplicative_Associativity}} \\
&= ((-1) \cdot x) \cdot x
&& \text{Theorem \ref{Theorem:Multiplicative_Commutativity}} \\
&= (-1) \cdot (x \cdot x)
&& \text{Theorem \ref{Theorem:Multiplicative_Associativity}} \\
&= - (x \cdot x)
&& \text{Definition \ref{Definition:Additive_Inverse}} \\
\end{align*}
Reduction:
\begin{align*}
\exp(\ln(-1) + \ln(x) + \ln(x))
\end{align*}

\section{Theorem:Multiplication of Negative and Negative}
\label{Theorem:Multiplication_of_Negative_and_Negative}
For $x \neq -\infty$,
\begin{align*}
(-x) \cdot (-x)
&= (-x) \cdot (-x) + 0
&& \text{Axiom \ref{Axiom:Additive_Identity}} \\
&= (-x) \cdot (-x) + (-(x \cdot x)) + x \cdot x
&& \text{Theorem \ref{Theorem:Additive_Inverse}} \\
&= (-x) \cdot (-x) + (-x) \cdot x + x \cdot x
&& \text{Theorem \ref{Theorem:Multiplication_of_Positive_and_Negative}} \\
&= (-x) \cdot ((-x) +  x) + x \cdot x
&& \text{Axiom \ref{Axiom:Distributivity}} \\
&= (-x) \cdot 0 + x \cdot x
&& \text{Theorem \ref{Theorem:Additive_Inverse}} \\
&= 0 + x \cdot x
&& \text{Theorem \ref{Theorem:Multiplication_of_Zero}} \\
&= x \cdot x
&& \text{Axiom \ref{Axiom:Additive_Identity}} \\
\end{align*}
Auto Simplification:
\begin{align*}
\exp(\ln(-1) + \ln(-1) + x)
&= \exp(2 \cdot \tau \cdot i + \ln(-1) + x)
&& \text{Implementation \ref{Implementation:Logarithm_of_Negative_One_1}} \\
&= \exp(2 \cdot \tau \cdot i + 2 \cdot \tau \cdot i + x)
&& \text{Implementation \ref{Implementation:Logarithm_of_Negative_One_1}} \\
&= \exp(4 \cdot \tau \cdot i + x)
&& \text{Implementation \ref{Implementation:Summation_of_Term_1}} \\
&= \exp(x)
&& \text{Implementation \ref{Implementation:Exponentiation_of_tau_i}} \\
\end{align*}

\section{Definition:Three}
\label{Definition:Three}
\begin{align*}
3 &= 2+1 \\
\end{align*}

\section{Definition:Four}
\label{Definition:Four}
\begin{align*}
4 &= 3+1 \\
\end{align*}

\section{Theorem:Two Plus Two}
\label{Theorem:Two_Plus_Two}
\begin{align*}
4
&= 3+1
&& \text{Definition \ref{Definition:Four}} \\
&= 2+1+1
&& \text{Definition \ref{Definition:Three}} \\
&= 2+2
&& \text{Definition \ref{Definition:Two}} \\
\end{align*}

\section{Theorem:Exponentiation of tau and i}
\label{Theorem:Exponentiation_of_tau_i}
\begin{align*}
\exp (4 \cdot \tau \cdot i + x)
&= \exp ((2+2) \cdot \tau \cdot i + x)
&& \text{Theorem \ref{Theorem:Two_Plus_Two}} \\
&= \exp(2 \cdot \tau \cdot i + 2 \cdot \tau \cdot i + x)
&& \text{Axiom \ref{Axiom:Distributivity}} \\
&= \exp(\ln(-1) + 2 \cdot \tau \cdot i + x)
&& \text{Theorem \ref{Theorem:Logarithm_of_Negative_One}} \\
&= \exp(\ln(-1) + \ln(-1) + x)
&& \text{Theorem \ref{Theorem:Logarithm_of_Negative_One}} \\
&= \exp(\ln(-1) + \ln(-1) + \ln(\exp(x)))
&& \text{Axiom \ref{Axiom:Inverse_Function_2}} \\
&= (-1) \cdot (-1) \cdot \exp(x)
&& \text{Definition \ref{Definition:Multiplication}} \\
&= 1 \cdot 1 \cdot \exp(x)
&& \text{Theorem \ref{Theorem:Multiplication_of_Negative_and_Negative}} \\
&= 1 \cdot \exp(x)
&& \text{Theorem \ref{Theorem:Multiplicative_Identity}} \\
&= \exp(x)
&& \text{Theorem \ref{Theorem:Multiplicative_Identity}} \\
\end{align*}
Simplification:
\begin{align}
\exp(4 \cdot \tau \cdot i + x) &= \exp(x) \label{Implementation:Exponentiation_of_tau_i}
\end{align}

\section{Theorem:Imaginary Unit}
\label{Theorem:Imaginary_Unit}
\begin{align*}
i \cdot i
&= \exp(\ln(i) + \ln(i))
&& \text{Definition \ref{Definition:Multiplication}} \\
&= \exp(\tau \cdot i + \ln(i))
&& \text{Theorem \ref{Theorem:Logarithm_of_i}} \\
&= \exp(\tau \cdot i + \tau \cdot i)
&& \text{Theorem \ref{Theorem:Logarithm_of_i}} \\
&= \exp((1+1) \cdot \tau \cdot i)
&& \text{Axiom \ref{Axiom:Distributivity}} \\
&= \exp(2 \cdot \tau \cdot i)
&& \text{Definition \ref{Definition:Two}} \\
&= -1
&& \text{Axiom \ref{Axiom:Exponentiation_of_tau_and_i}} \\
\end{align*}
Auto Simplification:
\begin{align*}
\exp(\ln(i) + \ln(i) + x)
&= \exp(\tau \cdot i + \ln(-1) + x)
&& \text{Implementation \ref{Implementation:Logarithm_of_i_1}} \\
&= \exp(\tau \cdot i + \cdot \tau \cdot i + x)
&& \text{Implementation \ref{Implementation:Logarithm_of_i_1}} \\
&= \exp(2 \cdot \tau \cdot i + x)
&& \text{Implementation \ref{Implementation:Summation_of_Term_1}} \\
&= \exp(\ln(-1) + x)
&& \text{Implementation \ref{Implementation:Logarithm_of_Negative_One_2}} \\
\end{align*}

\section{Theorem:Exponential Law}
\label{Theorem:Exponential_Law}
For $x \neq -\infty$ and $y \neq -\infty$,
\begin{align*}
\exp(x) \cdot \exp(y)
&= \exp(\ln(\exp(x)) + \ln(\exp(y)))
&& \text{Definition \ref{Definition:Multiplication}} \\
&= \exp(x + \ln(\exp(y)))
&& \text{Axiom \ref{Axiom:Inverse_Function_2}} \\
&= \exp(x + y)
&& \text{Axiom \ref{Axiom:Inverse_Function_2}} \\
\end{align*}
Reduction:
\begin{align*}
\exp(x + y)
\end{align*}

\section{Theorem:Logarithmic Law}
\label{Theorem:Logarithmic_Law}
For $x \neq -\infty$ and $y \neq -\infty$,
\begin{align*}
\exp(\ln(x \cdot y) + z)
&= \exp(\ln(\exp(\ln(x) + \ln(y))) +z)
&& \text{Definition \ref{Definition:Multiplication}} \\
&= \exp(\ln(x) + \ln(y) +z)
&& \text{Axiom \ref{Axiom:Inverse_Function_2}} \\
\end{align*}
Reduction:
\begin{align*}
\exp(\ln(x) + \ln(y) +z)
\end{align*}

\end{document}
